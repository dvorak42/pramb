\documentclass{article}

\usepackage{url,listings}
\lstset{
  language=Lisp,
  basicstyle=\ttfamily,
  showstringspaces=false}

\title{\texttt{ambc}: the most general \texttt{amb}}
\author{Jacob Hurwitz \and David Lawrence \and Steven Valdez}

\begin{document}

\maketitle

\begin{center}
  Source code is available at \url{http://github.com/dvorak42/pramb}.
\end{center}

\section{Introduction and motivation}

The \texttt{amb} function offers great adaptability by providing a
clean mechanism for programs to make choices by implicit backtracking.
We aimed to extend \texttt{amb} so that it might also represent
decisions over an undetermined or possibly infinite space of
alternatives.

We elected to 

\section{Implementation of \texttt{ambc}}

\section{\texttt{ambc} test cases}

\section{Applications to probability}
Once we implemented \texttt{ambc}, we were then able to apply it to probability
distributions in order to begin selecting over the values of the probability
distribution. In order to manipulate and deal with probability objects,
\textit{probobjs}, we constructed methods to transform these objects and combine
them together:

\begin{itemize}
  \item \textbf{p:sum}: Returns a probobj that is the sum of the input probobjs.
  \item \textbf{p:mult}: Returns a probobj that is the product of the input probobjs.
  \item \textbf{p:scale}: Returns a probobj that is the scaled form of the input.
  \item \textbf{p:shift}: Returns a probobj that is a shifted version of the
    input.
\end{itemize}

Once we had established a set of operations to perform on the probobjs, we are
then able to generate more complex probability distributions from the initial
uniform probability distribution. In order to help ensure the probability
distributions are correct, and to visualize the output of the probobjs, we also
implemented a ``p:display'' function that would generate a visual representation
of the probobj by taking many samples which are then placed into bins across the
distribution.

In addition to the methods that we created to manipulate probobjs, we are also
able to use our \texttt{ambc} construction to generate probobjs deriving from
the input one. For example, we can generate the part of the normal distribution
above $0.25$ by using:\\

\begin{lstlisting}
(define (px s f) 
  (let ((p (ambc p:normal))) 
    (require (> p 0.25))
    (s p)))
\end{lstlisting}

TODO ABOUT MONTE-CARLO STUFF.
\section{Directions for future work}

\end{document}
