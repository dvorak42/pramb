\documentclass[14pt]{beamer}
\mode<presentation>{\usetheme{Boadilla}}
\usepackage[english]{babel}
\usepackage{listings}
\lstset{
  language=Lisp,
  basicstyle=\ttfamily,
  showstringspaces=false}
\setbeamertemplate{navigation symbols}{}

\title{\texttt{ambc}: the most general \texttt{amb}}
\author[dlaw, dvorak42, jhurwitz]
  {Jacob Hurwitz \and David Lawrence \and Steven Valdez}
\date{May 13, 2013}

\begin{document}

\titlepage

\begin{frame}[fragile]{Motivation}
 Let's use \texttt{amb} to factor the number 22.

\begin{lstlisting}
(define (amb-count-up n)
  (amb n (amb-count-up (+ n 1))))
\end{lstlisting}
\begin{lstlisting}
(let ((a (amb-count-up 1))
      (b (amb-count-up 1)))
   (if (not (= (* a b) 22)) (amb))
   (list a b))
\end{lstlisting}

\pause

It doesn't work\ldots why?

\pause

  \begin{itemize}
  \item Call-by-value evaluation strategy
    \pause
  \item Backtracking by depth-first search
  \end{itemize}
\end{frame}

\begin{frame}[fragile]{Our plan}
  \begin{itemize}
  \item Use embedded interpreter from problem set 4
    \pause
  \item Implement backtracking by breadth-first search
    \pause
  \item Maintain call-by-value evaluation strategy
    \pause
  \item Replace \texttt{amb} with \texttt{ambc} (``\texttt{amb}-continuation'')
  \end{itemize}
\end{frame}

\begin{frame}[fragile]{\texttt{ambc} semantics}
\begin{lstlisting}
(ambc
  (lambda (succeed fail)
    (succeed 22)))
\end{lstlisting}
evaluates to 22 forever.

\vfill

\begin{lstlisting}
(ambc
  (lambda (succeed fail)
    (fail)))
\end{lstlisting}
fails immediately.
\end{frame}

\begin{frame}[fragile]{Maintaining state}
  The environment of the continuation passed to \texttt{ambc} is not
  reset during backtracking.
  \begin{lstlisting}
(define (succeed-once val)
  (define flag #t)
  (ambc
   (lambda (succeed fail)
     (if flag
         (begin
           (set! flag #f)
           (succeed val))
         (fail)))))
  \end{lstlisting}
\end{frame}

\begin{frame}[fragile]{Useful \texttt{ambc} examples}
  \begin{lstlisting}
(define (amb . alts)
  (ambc
   (lambda (succeed fail)
     (if (null? alts)
         (fail)
         (let ((result (car alts)))
           (set! alts (cdr alts))
           (succeed result))))))
  \end{lstlisting}
\pause
  \begin{lstlisting}
(define (amb-range low high)
  (ambc
   (lambda (succeed fail)
     (succeed (random-num low high))))
  \end{lstlisting}
\end{frame}

\begin{frame}[fragile]{Interpreter implementation}
  Based on the analyzing interpreter from problem set 4. \pause
  \begin{itemize}
  \item We pass around success continuations instead of returning the
    result. \pause
  \item We do not pass around failure continuations, instead storing
    them in a global queue. \pause
  \item We never explicitly pass around environments. \pause
  \end{itemize}
  \begin{lstlisting}
(define (analyze exp)
  (let (...)  ; compilation
    (lambda (succeed)
       ; evaluate exp
       (succeed value))))
  \end{lstlisting}
\end{frame}

\begin{frame}[fragile]{Environments}
  We store all environments in a central data structure which supports
  operations for backtracking:
  \begin{itemize}
  \item \texttt{reset-env-state}
  \item \texttt{grab-env-state}
  \item \texttt{restore-env-state}
  \end{itemize} \pause
  as well as the interfaces required by the evaluator:
  \begin{itemize}
    \item \texttt{current-env}, \texttt{push-env!}, \texttt{pop-env!}
    \item \texttt{get-proc-env}, \texttt{set-proc-env!}.
  \end{itemize} \pause
  \vfill Efficient implementation of the environment data structure is the
  focus of ongoing development work.
\end{frame}

\begin{frame}[fragile]{Using environments}
  \begin{lstlisting}
(defhandler execute-application
  (lambda (proc args succeed)
    (push-env! 
     (extend-environment
      (procedure-parameters proc)
      args
      (get-proc-env proc)))
    ((procedure-body proc)
     (lambda (val)
       (pop-env!)
       (succeed val))))
  compound-procedure?)
  \end{lstlisting}
\end{frame}

\begin{frame}[fragile]{Failure queue}
  \begin{lstlisting}
(define fail-queue)   ; initialized
(define global-fail)  ; by the repl

(define (add-branch cont)
  (enqueue! fail-queue
    (cons cont (grab-env-state))))
(define (fail)
  (if (queue-empty? fail-queue)
      (global-fail)
      (let ((p (dequeue! fail-queue)))
        (restore-env-state (cdr p))
        ((car p)))))
  \end{lstlisting}
\end{frame}

\begin{frame}[fragile]{\texttt{ambc}}
  \begin{lstlisting}
(define (analyze-ambc exp)
  (let ((fproc (analyze (cadr exp))))
    (lambda (succeed)
      (fproc
       (lambda (proc)
         (let loop ()     
           (execute-application proc
            (list (lambda (r)
                    (pop-env!)
                    (add-branch loop)
                    (succeed r))
                  fail)
            'dummy-succeed)))))))
  \end{lstlisting}
\end{frame}

\begin{frame}{Probability Objects}
  As seen above, we use the concept of ``Probability Objects'', functions which
  take in a success and failure continuation, along with functions that can
  operate on these objects:\\
  \begin{itemize}
    \item \texttt{p:sum P1 P2} - Returns a distribution that is the sum of the
        two original distributions P1 and P2.
    \item \texttt{p:scale P1 C} - Returns a distribution scaled by C.
    \item \texttt{p:mult P1 P2} - Returns the product of two distributions.
    \item \texttt{p:uniform} - A distribution representing a uniform number
      between 0 and 1.
    \item \texttt{p:normal} - A psuedo-normal distribution generated by the
      central limit theorem.
  \end{itemize}
\end{frame}

\begin{frame}{Probability Objects}
  In addition to these manipulations of the distributions, we also have
  functions to display them in different forms:\\
  \begin{itemize}
    \item \texttt{p:value N P} - Returns N samples from the distribution P.
    \item \texttt{p:display-samples P N} - Displays N samples of P in graph form.
    \item \texttt{p:display P} - \texttt{p:display-samples P 1000}
  \end{itemize}
\end{frame}

\begin{frame}{Examples}
  
\end{frame}

\begin{frame}{Ongoing work}
  
\end{frame}
\end{document}